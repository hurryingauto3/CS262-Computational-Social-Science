\documentclass[a4paper, 10pt]{article}
\usepackage[a4paper, total={6in, 8in}]{geometry}
\usepackage[colorlinks=true,linkcolor=black,anchorcolor=black,citecolor=black,filecolor=black,menucolor=black,runcolor=black,urlcolor=black]{hyperref}
\title{CS/SDP 262 Final Project Report}
\date{\today}
\author{Badvi, B.; Hamza, A.; Usaid, M.}

\begin{document}
\maketitle
\newpage
\tableofcontents
\newpage
\section {Introduction}
\subsection{Abstract}
Anthropocentrism and the capitalistic commodification of resources has led humans down a spiral of self-destruction, and equally increased a general disregard for nature. For most nature is an external entity, unbeknown to them that humanity is not exclusive from it. Following from this stems the negligence of intelligence shown in nature however, as we shall be exploring in our project, signs of intelligence can be found everywhere in nature. Our project deals with modelling the semi-intelligent behavior shown by an acellular organism known as Physarum Polycephalum or more commonly as Yellow slime mold. This organism has baffled biologist and researcher over its exhibition of intelligence besides the fact that it does not have a central nervous system. Surely without a central processing unit how can an organism function let alone solve difficult problems? Intriguingly enough P.Polycephalum cannot only perform basic life function but is able to solve many complex problems such as path optimization (Nakagaki et al., 2000). Research suggest that a pulsating flow of biochemicals within its tube-like structure handles P.Polycephalum’s intricate mobility and that the control of this fluid is the cause of its coordinated growth (Alim et al., 2013). Another research suggests that P.Polycephalum makes use of an external ‘spatial’ memory by using pheromones to mark visited regions (Reid et al., 2012).
\subsection{Storyline}
\subsection{Purpose}
\subsection{Approach}

\section {Details of Simulation Model}
\subsection{Agent Based Model (Minimum Food Distance)}
\subsubsection*{Agents}
\subsubsection*{Environment}
\subsubsection*{Learning}
\subsubsection*{Interactions}
\subsubsection*{Time}
\subsubsection*{Variables and Constants}
\subsubsection*{Processes}
\subsubsection*{Stochastic Measures}
\subsubsection*{Actions}
\subsubsection*{Inputs}
\subsubsection*{Outputs}

\subsection{Agent Based Model (Minimum Spanning Network)}
\subsubsection* {Agents}
\subsubsection* {Environment}
\subsubsection* {Learning}
\subsubsection* {Interactions}
\subsubsection* {Time}
\subsubsection* {Variables and Constants}
\subsubsection* {Processes}
\subsubsection* {Stochastic Measures}
\subsubsection* {Actions}
\subsubsection* {Inputs}
\subsubsection* {Outputs}

\section{Analysis of Data}
\section{Limitations of the Model}
\section{Conclusions}
\newpage
\section{Bibliography}

% \begin{enumerate}
%     \item Alim, K., Amselem, G., Peaudecerf, F., Brenner, M., & Pringle, A. (2013). Random network peristalsis in Physarum polycephalum organizes fluid flows across an individual. Proceedings of the National Academy of Sciences of the United States of America, 110(33), 13306-13311. Retrieved September 12, 2020, from \\ \url{http://www.jstor.org/stable/42712903}
%     \item Alim, K., Andrew, N., Pringle, A., Brenner, M. (2017). Mechanism of signal propagation in Physarum polycephalum. Proceedings of the National Academy of Sciences. Retrieved September 12, 2020, from \\\url{https://www.pnas.org/content/114/20/5136.short.}

%     \item Nakagaki, T., Yamada, H. & Tóth, Á. Maze-solving by an amoeboid organism. Nature 407, 470 (2000). \url{https://doi.org/10.1038/35035159}
    
%     \item Reid, C., Latty, T., Dussutour, A., & Beekman, M. (2012). Slime mold uses an externalized spatial "memory" to navigate in complex environments. Proceedings of the National Academy of Sciences of the United States of America, 109(43), 17490-17494. Retrieved September 12, 2020, from \url{http://www.jstor.org/stable/41829697}
    
%     \item Tero, A., Takagi, S., Saigusa, T., Ito, K., Bebber, D., Flicker, M., . . . Nakagaki, T. (2010). Rules for Biologically Inspired Adaptive Network Design. Science, 327(5964), new series, 439-442. Retrieved September 13, 2020, from \url{http://www.jstor.org/stable/40508592}
% \end{enumerate}
\end{document}